\documentclass[]{article}
\usepackage{lmodern}
\usepackage{amssymb,amsmath}
\usepackage{ifxetex,ifluatex}
\usepackage{fixltx2e} % provides \textsubscript
\ifnum 0\ifxetex 1\fi\ifluatex 1\fi=0 % if pdftex
  \usepackage[T1]{fontenc}
  \usepackage[utf8]{inputenc}
\else % if luatex or xelatex
  \ifxetex
    \usepackage{mathspec}
    \usepackage{xltxtra,xunicode}
  \else
    \usepackage{fontspec}
  \fi
  \defaultfontfeatures{Mapping=tex-text,Scale=MatchLowercase}
  \newcommand{\euro}{€}
\fi
% use upquote if available, for straight quotes in verbatim environments
\IfFileExists{upquote.sty}{\usepackage{upquote}}{}
% use microtype if available
\IfFileExists{microtype.sty}{%
\usepackage{microtype}
\UseMicrotypeSet[protrusion]{basicmath} % disable protrusion for tt fonts
}{}
\usepackage[margin=1in]{geometry}
\ifxetex
  \usepackage[setpagesize=false, % page size defined by xetex
              unicode=false, % unicode breaks when used with xetex
              xetex]{hyperref}
\else
  \usepackage[unicode=true]{hyperref}
\fi
\hypersetup{breaklinks=true,
            bookmarks=true,
            pdfauthor={Yago Durán Cid},
            pdftitle={Applied Logistic Regression - Exercise Week 3},
            colorlinks=true,
            citecolor=blue,
            urlcolor=blue,
            linkcolor=magenta,
            pdfborder={0 0 0}}
\urlstyle{same}  % don't use monospace font for urls
\setlength{\parindent}{0pt}
\setlength{\parskip}{6pt plus 2pt minus 1pt}
\setlength{\emergencystretch}{3em}  % prevent overfull lines
\setcounter{secnumdepth}{0}

%%% Use protect on footnotes to avoid problems with footnotes in titles
\let\rmarkdownfootnote\footnote%
\def\footnote{\protect\rmarkdownfootnote}

%%% Change title format to be more compact
\usepackage{titling}

% Create subtitle command for use in maketitle
\newcommand{\subtitle}[1]{
  \posttitle{
    \begin{center}\large#1\end{center}
    }
}

\setlength{\droptitle}{-2em}
  \title{Applied Logistic Regression - Exercise Week 3}
  \pretitle{\vspace{\droptitle}\centering\huge}
  \posttitle{\par}
  \author{Yago Durán Cid}
  \preauthor{\centering\large\emph}
  \postauthor{\par}
  \predate{\centering\large\emph}
  \postdate{\par}
  \date{30/05/2015}



\begin{document}

\maketitle


\textbf{WEEK 3}

\emph{Exercise 1:} Use the Myopia Study (MYOPIA.csv)

\begin{enumerate}
\def\labelenumi{\alph{enumi}.}
\itemsep1pt\parskip0pt\parsep0pt
\item
  Using the results from Week 2, Exercise 1, compute 95 percent
  confidence intervals for the slope coefficient SPHEQ. Write a sentence
  interpreting this confidence.
\end{enumerate}

We keep the basic model we used in homework 2. This is
$\pi(x)=E(y|x)=\frac{e^{(\beta_0+\beta_1SPHEQ)}}{1+e^{(\beta_0+\beta_1SPHEQ)}}$
which gives foolowing results:

\begin{verbatim}
## 
## Call:
## glm(formula = MYOPIC ~ SPHEQ, family = "binomial", data = data)
## 
## Deviance Residuals: 
##     Min       1Q   Median       3Q      Max  
## -1.6435  -0.4533  -0.2681  -0.1029   3.1602  
## 
## Coefficients:
##             Estimate Std. Error z value Pr(>|z|)    
## (Intercept)  0.05397    0.20675   0.261    0.794    
## SPHEQ       -3.83310    0.41837  -9.162   <2e-16 ***
## ---
## Signif. codes:  0 '***' 0.001 '**' 0.01 '*' 0.05 '.' 0.1 ' ' 1
## 
## (Dispersion parameter for binomial family taken to be 1)
## 
##     Null deviance: 480.08  on 617  degrees of freedom
## Residual deviance: 337.34  on 616  degrees of freedom
## AIC: 341.34
## 
## Number of Fisher Scoring iterations: 6
\end{verbatim}

As we know, the confidence interval can be estimated as
$\beta_j \pm z_{\frac{1-\alpha}{2}}\hat{\sigma_{\beta_j}}$

Thus, the higher bound of the confidence interval is
$-3.83310 + 1.96\sqrt{0.17503316}$=-3.0130931\\And the lower bound is
$-3.83310 - 1.96\sqrt{0.17503316}$=-4.6531021

Given the confidence interval estimated above we can say that, with 95\%
confidence, the true value of $\beta_{SPHEQ}$ is between -3.0130931 and
-4.6531021 and, therefore, is different than zero (i.e.: the value of
SPHEQ impacts the probability of MYOPIA being 1 or 0)

\begin{enumerate}
\def\labelenumi{\alph{enumi}.}
\setcounter{enumi}{1}
\itemsep1pt\parskip0pt\parsep0pt
\item
  Use R to obtain the estimated covariance matrix. Compute the logit and
  estimated logistic probability for a subject with SPHEQ = 2. Evaluate
  the endpoints of the 95 percent confidence intervals for the logit and
  estimated logistic probability. Write a sentence interpreting the
  estimated probability and its confidence interval.
\end{enumerate}

The variance-covariance matrix of the model is:

\begin{verbatim}
##             (Intercept)       SPHEQ
## (Intercept)  0.04274486 -0.06337768
## SPHEQ       -0.06337768  0.17503316
\end{verbatim}

Based on the estimated model, we can obtain the logit for SPHEQ=2

\begin{verbatim}
##    estimate Std.Error
## 1 -7.612222 0.6995475
\end{verbatim}

Substituting in the probability fucntion
$Prob(SPHEQ=2)=\frac{e^{0.0004941278\pm1.96*0.0003454951}}{1+e^{0.0004941278\pm1.96*0.0003454951}}$

Average probability of MYOPIA=1 given that SPHEQ=2 is 0.0494128\%\\At
95\% confidence, the probability of MYOPIA=1 given that SPHEQ=2 is
between 0.1943908\% and 0.0125468\%

\emph{Exercise 2:} Use the ICU study (icu.csv) a. Using the results from
Week 1, Exercise 2, part (d), compute 95 percent confidence intervals
for the slope and constant term. Write a sentence interpreting the
confidence interval for the slope.

As in previous section we show the results of model estimated.

\begin{verbatim}
## 
## Call:
## glm(formula = STA ~ AGE, family = "binomial", data = icu)
## 
## Deviance Residuals: 
##     Min       1Q   Median       3Q      Max  
## -0.9536  -0.7391  -0.6145  -0.3905   2.2854  
## 
## Coefficients:
##             Estimate Std. Error z value Pr(>|z|)    
## (Intercept) -3.05851    0.69608  -4.394 1.11e-05 ***
## AGE          0.02754    0.01056   2.607  0.00913 ** 
## ---
## Signif. codes:  0 '***' 0.001 '**' 0.01 '*' 0.05 '.' 0.1 ' ' 1
## 
## (Dispersion parameter for binomial family taken to be 1)
## 
##     Null deviance: 200.16  on 199  degrees of freedom
## Residual deviance: 192.31  on 198  degrees of freedom
## AIC: 196.31
## 
## Number of Fisher Scoring iterations: 4
\end{verbatim}

The confidence interval at 95\% probablity for both the intercept and
the variable AGE is:

\begin{verbatim}
##                   2.5 %      97.5 %
## (Intercept) -4.42280739 -1.69421908
## AGE          0.00683723  0.04824799
\end{verbatim}

\begin{enumerate}
\def\labelenumi{\alph{enumi}.}
\setcounter{enumi}{1}
\itemsep1pt\parskip0pt\parsep0pt
\item
  Obtain the estimated covariance matrix for the model fit from Week 1,
\end{enumerate}

\begin{verbatim}
##              (Intercept)           AGE
## (Intercept)  0.484529087 -0.0071029945
## AGE         -0.007102994  0.0001116015
\end{verbatim}

\begin{enumerate}
\def\labelenumi{\alph{enumi}.}
\setcounter{enumi}{3}
\itemsep1pt\parskip0pt\parsep0pt
\item
  Compute the logit and estimated logistic probability for a 60-year old
  subject. Compute a 95 percent confidence intervals for the logit and
  estimated logistic probability. Write a sentence or two interpreting
  the estimated probability and its confidence interval.
\end{enumerate}

\begin{verbatim}
##    estimate Std.Error
## 1 -1.405957 0.1842154
\end{verbatim}

Given that the value of the logit is -1.4059567 with standar error
0.1842154 we can estimate the probability as we did in first section.

Average probability of STA=1 given that AGE=60 is 19.6872578\%

At 95\% confidence, the probability of STA=1 given that AGE=60 is
between 26.0206709\% and 14.5913451\%

\emph{Exercise 3:}\\Use the ICU study (icu.csv)\\Use the ICU data and
consider the multiple logistic regression model of vital status, STA, on
age (AGE), cancer part of the present problem (CAN), CPR prior to ICU
admission (CPR), infection probable at ICU admission (INF), and race
(RACE).\\a. The variable RACE is coded at three levels. Prepare a table
showing the coding of the two design variables necessary for including
this variable in a logistic regression model.

In the original dataset, RACE variable is included as integer, which is
incorrect

\begin{verbatim}
##    Min. 1st Qu.  Median    Mean 3rd Qu.    Max. 
##   1.000   1.000   1.000   1.175   1.000   3.000
\end{verbatim}

We have to convert RACE variable from integer to factor

\begin{verbatim}
##   1   2   3 
## 175  15  10
\end{verbatim}

\begin{enumerate}
\def\labelenumi{\alph{enumi}.}
\setcounter{enumi}{1}
\itemsep1pt\parskip0pt\parsep0pt
\item
  Write down the equation for the logistic regression model of STA on
  AGE, CAN, CPR, INF, and RACE. Write down the equation for the logit
  transformation of this logistic regression model. How many parameters
  does this model contain?
\end{enumerate}

$\pi(x)=\frac{e^{\beta_0+\beta_{1}AGE+\beta_{2}CAN+\beta_{3}CPR+\beta_{4}INF+\beta_{5}RACE2+\beta_{6}RACE3}}{1+e^{\beta_0+\beta_{1}AGE+\beta_{2}CAN+\beta_{3}CPR+\beta_{4}INF+\beta_{5}RACE2+\beta_{6}RACE3}}$

The logit is

$g(x)=\beta_0+\beta_{1}AGE+\beta_{2}CAN+\beta_{3}CPR+\beta_{4}INF+\beta_{5}RACE2+\beta_{6}RACE3$

We have, thus, seven parameters to estimate.

\begin{enumerate}
\def\labelenumi{\alph{enumi}.}
\setcounter{enumi}{2}
\itemsep1pt\parskip0pt\parsep0pt
\item
  Write down an expression for the likelihood and log likelihood for the
  logistic regression model in part (b). How many likelihood equations
  are there? Write down an expression for a typical likelihood equation
  for this problem.
\end{enumerate}

The likelihood would be as
follows:\\$\ell(\beta)=\Pi_{i=1}^{n}\pi(x_i)^{y_i}(1-\pi(x_i)^{1-y_i}$

Where $y_i=1$ if STA=1 and $y_i=0$ otherwise.

Unsing logarithms we get thee log
likelihood:\\$log(\ell(\beta))=\sum_{i=1}^n((y_ilog(\pi(x_i))+(1-y_i)log(1-\pi(x_i)))$

\begin{enumerate}
\def\labelenumi{\alph{enumi}.}
\setcounter{enumi}{3}
\itemsep1pt\parskip0pt\parsep0pt
\item
  Using a logistic regression package, obtain the maximum likelihood
  estimates of the parameters of the logistic regression model in part
  (b). Using these estimates write down the equation for the fitted
  values, that is, the estimated logistic probabilities.
\end{enumerate}

The estimated parameters are as follows:

\begin{verbatim}
## 
## Call:
## glm(formula = STA ~ AGE + CAN + CPR + INF + RACE, family = "binomial", 
##     data = icu2)
## 
## Deviance Residuals: 
##     Min       1Q   Median       3Q      Max  
## -1.3703  -0.6823  -0.5421  -0.3082   2.5124  
## 
## Coefficients:
##             Estimate Std. Error z value Pr(>|z|)    
## (Intercept) -3.51152    0.81443  -4.312 1.62e-05 ***
## AGE          0.02712    0.01159   2.340  0.01926 *  
## CAN          0.24451    0.61681   0.396  0.69180    
## CPR          1.64650    0.62341   2.641  0.00826 ** 
## INF          0.68067    0.38042   1.789  0.07357 .  
## RACE2       -0.95708    1.08445  -0.883  0.37748    
## RACE3        0.25975    0.87127   0.298  0.76561    
## ---
## Signif. codes:  0 '***' 0.001 '**' 0.01 '*' 0.05 '.' 0.1 ' ' 1
## 
## (Dispersion parameter for binomial family taken to be 1)
## 
##     Null deviance: 200.16  on 199  degrees of freedom
## Residual deviance: 179.30  on 193  degrees of freedom
## AIC: 193.3
## 
## Number of Fisher Scoring iterations: 5
\end{verbatim}

Which translates in

$\pi(x_i)=\frac{e^{-3.51152+0.02712AGE_i+0.24451CAN_i+1.64650CPR_i+0.68067INF_i-0.95708RACE2_i+0.25975RACE3_i}}{1+e^{-3.51152+0.02712AGE_i+0.24451CAN_i+1.64650CPR_i+0.68067INF_i-0.95708RACE2_i+0.25975RACE3_i}}$

\begin{enumerate}
\def\labelenumi{\alph{enumi}.}
\setcounter{enumi}{4}
\itemsep1pt\parskip0pt\parsep0pt
\item
  Using the results of the output from the logistic regression package
  used in part (d), assess the significance of the slope coefficients
  for the variables in the model using the likelihood ratio test. What
  assumptions are needed for the p-values computed for this test to be
  valid? What is the value of the deviance for the fitted model?
\end{enumerate}

In the likelihood test the null hypotheis is: $H_0: \forall\beta_i=0$

$G=Deviance_{model\,without\,variables}-Deviance_{model\,with\,variables}\,\rightsquigarrow\chi^2_{df}$

In our case, G=200.1609694-179.3007274=20.860242

The p-value for the model taking a $\chi^{2}_{df=6}=$ 0.0019437 Being
the p-value\textless{}0.05 we reject the null hypothesis.

\begin{enumerate}
\def\labelenumi{\alph{enumi}.}
\setcounter{enumi}{5}
\itemsep1pt\parskip0pt\parsep0pt
\item
  Use the Wald statistics to obtain an approximation to the significance
  of the individual slope coefficients for the variables in the model.
  Fit a reduced model that eliminates those variables with
  nonsignificant Wald statistics. Assess the joint (conditional)
  significance of the variables excluded from the model. Present the
  results of fitting the reduced model in a table.
\end{enumerate}

Reviewing the significance obtained fr parameters in section d we can
state that variables CAN and RACE are not significant. AGE and CPR are
significant at 95\% confidence while INF is open to debate. Below we
show details for the model inlcluding AGE, CPR and INF variables.

\begin{verbatim}
## 
## Call:
## glm(formula = STA ~ AGE + CPR + INF, family = "binomial", data = icu2)
## 
## Deviance Residuals: 
##     Min       1Q   Median       3Q      Max  
## -1.3633  -0.6810  -0.5524  -0.3091   2.4868  
## 
## Coefficients:
##             Estimate Std. Error z value Pr(>|z|)    
## (Intercept) -3.57604    0.77306  -4.626 3.73e-06 ***
## AGE          0.02792    0.01136   2.458  0.01397 *  
## CPR          1.63066    0.61553   2.649  0.00807 ** 
## INF          0.69708    0.37750   1.847  0.06481 .  
## ---
## Signif. codes:  0 '***' 0.001 '**' 0.01 '*' 0.05 '.' 0.1 ' ' 1
## 
## (Dispersion parameter for binomial family taken to be 1)
## 
##     Null deviance: 200.16  on 199  degrees of freedom
## Residual deviance: 180.51  on 196  degrees of freedom
## AIC: 188.51
## 
## Number of Fisher Scoring iterations: 5
\end{verbatim}

In order to furtherr assess the significance of INF (which p-value
remains higher than 0.05 but lower than 0.1) we can perform a likelihood
test comparing the model with no INF to the model including INF.

G=200.1609694-180.5134271=3.4392532

The p-value for INF taking a $\chi^{2}_{df=1}=$ 0.0636645 Being the
p-value\textgreater{}0.05 we can discard INF variable at 95\%
confidence.

\begin{enumerate}
\def\labelenumi{\alph{enumi}.}
\setcounter{enumi}{6}
\itemsep1pt\parskip0pt\parsep0pt
\item
  Using the results from part (f), compute 95 percent confidence
  intervals for all coefficients in the model. Write a sentence
  interpreting the confidence intervals for the non-constant covariates.
\end{enumerate}

The 95\% confidence interval for the parameters estimated in our last
and shortest models are:

\begin{verbatim}
##                    2.5 %      97.5 %
## (Intercept) -4.813107596 -1.89080341
## AGE          0.007755898  0.05145892
## CPR          0.593811578  2.97437261
\end{verbatim}

\end{document}
